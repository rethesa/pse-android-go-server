
\subsection{Clientseitige Abweichungen}

\textbf{communication}

package: *.client.communication
\begin{enumerate}
	\item Alle Antworten vom Server konnten zu Response und ObjectResonse zusammengefasst werden. Der Vorteil von ObjectResponse ist, dass alle möglichen Objekte darin verpackt und verschickt werden können.
	\item Zum Serialisieren brauchten wir Wrapper Klassen, weshalb es nun mehrere Serializable Klassen gibt. Dies war uns beim Entwurf noch nicht bewusst und ist erst in der Implementierung aufgefallen.
\end{enumerate}


\textbf{controller}

package: *.client.controller.database
\begin{enumerate}
	\item Da es weder FeedEntryAllocation noch FeedEntryAppointment mehr gibt, gibt es auch keinen AllocationService und keinen AppointmentService mehr (hießen davor ServiceAllocation und ServiceAppointment).
	\item GroupService und UserService haben noch ein paar weitere Methoden erhalten um das Arbeiten mit der android Datenbank zu erleichtern. Zum Beispiel deleteAll() um alle Informationen von der Tabelle zu löschen.
	\item insertData() in GroupService und UserService geben nicht mehr einen boolean zurück sondern sind nur void.
\end{enumerate}

package: *.client.controller.objectStructure
\begin{enumerate}
	\item In den Methoden von AccountHandler und GroupHandler müssen je die Activity in der die Methode aufgerufen werden als Methodenparameter mitgegeben werden, da man sonst keine Request starten kann.
	\item In GroupHandler kam ein neuer Constructor hinzu. Bei diesem werden alle Informationen einer bereits existierenden Gruppe umbergeben, damit man diese erstellen kann und die Argumente als Methodenparameter mitgibt.
\end{enumerate}


\textbf{model}

package: *.client.model.database
\begin{enumerate}
	\item FeedEntryAllocation und FeedEntryAppointment wurden entfernt und in FeedEntryUser und FeedEntryGroup integriert. So ist die Datenbank besser erweiterbar, da weniger Tabellen angepasst werden müssen. Die Standorte von Usern werden nicht mehr auf der android Datenbank gespeichert sondern wenn der GoStatus aktiv ist regelmäßig vom Server abgerufen.
	\item DBHelperAllocation, DBHelperAppointment und DBHelperUser existieren nicht mehr sondern es gibt nur noch DBHelperGroup. Das ist ein Fehler, der beim Entwurf nicht aufgefallen ist. Davor waren es mehrere Datenbanken mit je einer Tabelle. Jetzt ist es eine Datenbank mit mehreren Tabellen.
\end{enumerate}

package: *client.model.objectStructure
\begin{enumerate}
	\item Der GroupClient wird nicht mehr über eine groupId sondern nur über seinen Namen identifiziert.
	\item In SimpleUser wird im Constructor nicht mehr die deviceId sondern die userId übergeben, da die deviceId direk abgerufen wird und die userId vom Server übermittelt wird.
	\item UserDecoratorClient hat noch ein zusätzliches Attribut dafür erhalten, ob dasjenige Mitglied Admin oder nur GroupMember ist, um diese Information aus dem Object auslesen zu können um es in der Datenbank zu speichern.
	\item In GroupClient müssen allen Methoden die Activity mitgegeben werden, damit diese einen Request an den Server starten können. Die meisten anderen Methodenparameter sind dabei weggefallen. Die getMember() wurde zu getMemberType() geändert und gibt nun einen boolean statt einem UserDecoratorClient zurück.
	\item In AppointmentDate wurde das Format von Datum und Uhrzeit von Date zu String geändert, da so unnötige Typumwandlungen im Code vermieden werden können.
\end{enumerate}


\textbf{view}

package: *client.view

\begin{enumerate}
	\item Die Base Activity wurde komplett neu erstellt. Ist zuständig für den NavigationDrawer und alle Verknüpfungen, die dieser erstellt. Die Ausführung von diesem haben wir im Entwurf aus zeitlichen/gesundheitlichen Gründen nicht rechtzeitig kommuniziert bekommen. Die GroupActivity erbt von dieser.
	\item Die Methode showDatePickerDialog(Viwe) (analog für Time) des DatePickerFragment und TimePickerFragment sind ins GroupAppointmentFragment verschoben worden, wo sie nun private sind. Diese öffnen das jeweilige Fragment in dem Uhrzeit und Datum gewählt werden können.
	\item GroupActivity: onBackPressed():void, wurde @Overrid.Die Methode ist dafür da, wenn man in einem Fragment A ist und in einanderes Fragment B weitergeleitet wird, dass durch das klicken des zurück Kopfes man nicht in die letzte Activity kommt, sondern dass man von dem Fragment B wieder zurück in das Vorherige (hier A) kommt. Bezüglich dem Wechseln der Fragments gab es oft Schwierigkeiten, deshalb wurde diese Methode überschrieben.

	onStart():void, wurde @Override. Hier wurde die Response bezüglich dem beitreten eines Nutzers in eine Gruppe mittels klicken des Links.

	onDestroy():void, wurde @Override .In der onDestroy wird der Receiver für JoinGroup mit Link, also der Receiver der in der onStart Methode implementiert wird, wird hier geschlossen, damit dieser nicht andere Respones abfängt.
	\item In GroupAppointmentFragment, GroupMapGoFragment, GroupMembersFragment, GroupnameCreateFragment und UsernameRegistrationFragment wurden onAttach und onDetach hinzugefügt, da diese für die Reciever der Responses des Servers benötigt werden.
	\item GroupMapFragment:

	+group:GroupClient, wurde protected, damit die davon erbenden Fragments GroupMapGoFragment und GroupMapNotGoFragment auf die Variable zugreifen können, und diese verändern

	+setMyLocation(boolean):void, wurde hinzugefügt Wird in der GroupMapGoFragment gerufen, damit der eigene Standort ermittelt wird. Da der eigene Standort in die Karte eingezeichnet werden muss, muss die Methode in der GroupMapFragment implementiert werden und wird nur in der GroupMapGoFragment gerufen. Die Vererbung erlaubt es nicht, auf die Map in der Oberklasse zu bearbeiten, um somit das einzeichnen des Standortes zu vereinfachen und effektiv zu machen, haben wir uns auf diese Methode geeinigt.

	+setMyGroupmemberLocation(LinkedList<GpsObject>):void, wurde hinzugefügt. Die Methode implementiert die Möglichkeit eine Liste von GpsObjects in Clustern zusammenzufassen und diese dann als blaue Kreise mit einer Zahl, welche der Anzahl der zusammengefassten Standorte repräsentiert, in die Karte einzeichnet. Die Methode sorgt auch, dass die Punkte welche zusammengefasst sind, beim Rein und Raus Zoomen wieder zusammengefasst beziehungsweise getrennt in der Karte eingezeichnet werden.

	+onResume():void, wurde hinzugefügt Die Methode implementier nur, dass wenn man zurück in das Fragment kommt, die Karte von OpenStreetMap die default Einstellungen übernehmen soll, falls keine hinterlegt wurden.
	startService:void, wurde protected, damit das davon erbende Fragment GroupMapGoFragment diese implementieren kann
	\item GroupMapGoFragment
	startService():void, wurde protected, damit von diesem Fragment aus der GoIntentService gestartet werden kann
	\item GroupMembersFragment
	+onCreateOptionsMenu(Menu, MenuInflater):void, wurde hinzugefügt
	\item MemberAdapter
	extends RecyclerView, damit der Client nicht alle Mitglieder auf einmal laden muss
	+static class PersonViewHolder extends RecyclerView.ViewHolder Hier wird klar getrennt wie was formatiert wird. In dem RecyclerView kann man über CardView einstellen wie jedes Element formatiert und angezeigt wird. In dem Adapter werden sowohl die notwendigen Daten geholt, ausgetauscht und jedes Element formatiert.
	\item PlacePickerFragment extends Fragment
	wurde komplett hinzugefügt. Ursprünglicher Plan war es, diesen über das GroupMapFragment laufen zu lassen. Aber aus Gründen der stilistischen objektorientierten Implementierung, haben wir dafür eine neue Klasse erstellt
	\item UsernameActivity
	+onBackPressed():void, wurde hinzugefügt

\end{enumerate}

