\section{Server}


\subsection{Implementierte Funktionen des Servers}

\textbf{/sdjkfla/ Benutzerkonten verwalten}

Über \textit{RegistrationRequest} kann man ein neues Benutzerkonto auf dem Server anlegen.\\
Das Benutzerkonto wird an die Geräte-ID des Smartphones gebunden und in der\\
Datenbank des Servers hinterlegt.\\
Mit der \textit{DeleteUserRequest} kann der zuvor erstellte Account gelöscht werden.

\textbf{/sdjkfla/ Umbenennen des Benutzers}

Zusätzlich kann der Benutzer über die \textit{RenameUserRequest} seinen Nutzernamen ändern.

\textbf{/sdjkfla/ Gruppen erstellen / löschen}

Ein registrierter Benutzer kann via \textit{CreateGroupRequest} Gruppen auf dem\\
Server anlegen und per \textit{DeleteGroupRequest} wieder löschen.\\
Der Benutzer ist automatisch in der eigens erstellten Gruppe Admin.

\textbf{/sdjkfla/ Gruppe: Benutzer einladen / entfernen}

Mit der \textit{CreateLinkRequest} kann ein Gruppenadmin einen Einladungslink\\
erstellen. Das im Link enthaltene \textit{'Secret'} wird in der Gruppe auf\\
dem Server hinterlegt.\\
Über die \textit{KickMemberRequest} kann ein Gruppenmitglied aus der Gruppe\\
entfernt werden.

\textbf{/sdjkfla/ Gruppe: Mitglieder zu Administratoren machen}

Mit der \textit{MakeAdminRequest} können Mitglieder zu Admins befördert werden.

\textbf{/sdjkfla/ Gruppe: beitreten / verlassen}

Ein anderer Benutzer kann mit einer \textit{JoinGroupRequest} einer Gruppe beitreten.\\
Das zum Beitreten benutzer \textit{'Secret'} wird aus der Gruppe vom Server gelöscht.\\
Eine Gruppe kann mit der \textit{LeaveGroupRequest} verlassen werden.

\textbf{/sdjkfla/ Gruppe: Treffpunk festlegen}

Mit der \textit{SetAppointmentRequest} kann der Gruppenadmin einen Treffpunkt festlegen.
Dieser wird in der Server-Datenbank gespeichert, wo er von anderen Gruppenmitgliedern
abgerufen werden kann.


\textbf{/sdjkfla/ Gruppe: GPS Daten mitteilen / empfangen}

Per \textit{BroadcastGpsRequest} kann ein Gruppenmitglied der Gruppe seinen Standort
mitteilen. Gleichzeitig erhält er, verpackt in einer \textit{ObjectResponse},
die Standorte der anderen Gruppenmitglieder.

\textbf{/sdjkfla/ Gruppe: Umbenennen}

Analog zur \textit{RenameUserRequest}, kann man mit der \textit{RenameGroupRequest}\\
Den Gruppennamen ändern (der Name muss eindeutig sein).

\subsection{Abweichungen vom Entwurf}


