\section{Einleitung}

Die Anwendungslogik bildet die eigentliche Kommunikationsschnittstelle zur onlinen
Datenverwaltung. 
Dazu gibt die Anwendungslogik die Ein­/Ausgabedaten der View per Intent
an den NetworkIntentService weiter. Dieser kümmert sich darum, dass Anfragen vom Client
im Hintergrund bearbeitet und an den Server gesendet werden. Die Antworten vom Server
werden von Recievern auf dem Client empfangen. Ist dieser erfolgreich,
so speichert die Anwendungslogik auch die Daten in der Datenbank des Clients. Einzelne wichtige
Daten wie UserName, UserId und den Gruppenname der Gruppe in der man sich aktuell befindet werden
in SharedPreferences gespeichert und weitergegeben.

Auf der android Datenbank gibt es zwei Tabellen. Die eine speichert die Gruppen und den dazugehörigen
Treffpunkt und die andere die User und in welchen Gruppen diese Mitglied sind. Dabei werden nur die
Gruppen und zugehörigen Mitglieder gespeichert in denen der aktuelle User auch Mitglied ist.

Die Klassen der Anwendungslogik wurden komplett in JAVA implementiert.
Auf dem Client wurden die Klassen aufgeteilt in Model, View, Controller und Communication.
Auf dem Server wurden die Klassen aufgeteilt in blablabla hier muss Tarek was schreiben
Da es sich um eine objektorientierte Programmierung handelt, wurden die einzelnen Funktionen der
Anwendungslogik in geeignete Klassen aufgeteilt. So entstanden einige Klassen, die jede einen Teil
der Realisierung übernimmt.


Die Klassen der Anwedungslogik wurden in zwei Paketen gruppiert: *.al und *.beans  mit
jeweiligen Klassen: