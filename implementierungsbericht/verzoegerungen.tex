\section{Verzögerugen bei der Implementierung}

Zu Beginn der Implementierungsphase waren wir noch relativ gut im Zeitplan. Leider hat sich das relativ schnell in eine andere Richtung entwickelt.
Zum Einen hatten wir gehofft, dass Matthias sich um Hibernate kümmert und weitere Aufgaben auf dem Server übernimmt. Zum Anderen hatten wir auch erwartet, dass die View weniger Arbeit wäre und Android Studio weniger Probleme machen würde.
Bei ungefähr der Hälfte der Implementierungsphase wurde dann auch noch ein Gruppenmitglied krank und konnte die rechte Hand kaum noch verwenden und die Woche darauf wurde ein anderes Mitglied krank, ohne dass wir etwas neues von unserem fünften Gruppenmitglied wussten. Auch wenn wir uns nicht darauf verlassen wollten hatten wir natürlich gehofft, dass sich Matthias an unserem Projekt noch beteiligen würde.
Was wir in der letzten Woche total unterschätzt hatten war die Interaktion zwischen Server und Client. Da Matthias nun offiziell nicht mehr Mitglied unserer Gruppe ist, musste Hibernate noch implementiert werden und solange konnten wir auch die Reciever im Client nicht testen. Als wir soweit waren, dass die ersten Requests and den Server gesendet werden konnten ist uns aufgefallen, dass Reciever alle Responses abfangen die in die App eingehen. Das hat natürlich zu vielen Problemen geführt die leider nicht alle behoben werden konnten. 
Wir hatten für einen sehr großen und uns allen unbekannten Teil (also wie genau Client und Server interagieren) zu wenig Zeit, um ein ordentliches Ergebnis liefern zu können. Hat ein Teil funktioniert, hat ein anderer nicht mehr funktioniert und da es am Anfang hieß, dass Matthias sich um das Testen des Codes kümmern sollte, haben wir kaum Unit Test implementiert, was natürlich die Systeminteraktion erschwert. Fehler zu finden war sehr aufwändig. Und wenn der Server aktualisiert wurde konnte man solange natürlich nicht testen, ob die Reciever funktionieren und mit der Response umgehen können.

Also zusammengefasst lief es am Anfang ganz gut und wir kamen voran aber die einzelnen Teile die jeder für sich implementiert hat dann zu einem Großen zusammen zu fügen hat uns sehr viele Nerven und Zeit gekostet. 