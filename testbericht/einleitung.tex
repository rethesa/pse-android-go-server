Die Android-Go-App erleichtert es Nutzern sich in Gruppen zu organisieren und Treffpunkte zu vereinbaren. Sobald der Zeitpunkt des Treffens näher rückt, können alle Mitglieder ihre GPS-Daten an die anderen Gruppenmitglieder übermitteln und diese darüber informieren, wie weit sie sich vom Zielort entfernt befinden.

Dieses Dokument beschreibt dabei die Durchführung der Testphase der einzelnen Module und Applikationen, um eine möglichst störungsfreie Benutzung der Android-Go-App zu gewährleisten. Dabei werden auf die im Pflichtenheft definierten Testfälle und Szenarien eingegangen.

Zusätzlich wurden die in der Implementierungsphase in Verzug geratenen Unit Tests nachgeholt, als auch die Hauptfunktion (das Versenden, Empfangen, Aktualisieren und Anzeigen der GPS-Standorte in der Kartenansicht) der Android-Go-App implementiert. 
Funktionen wie Benutzer und Gruppen löschen oder eine Gruppe verlassen wurden auch noch nachträglich implementiert und sind im Allgemeinen funktionstüchtig. Im Gegensatz zu den anderen Funktionen der App sind sie aber noch fehleranfällig und nicht ausreichend getestet.
