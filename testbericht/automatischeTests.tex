\subsection{Unit Tests}
	\paragraph{Client Controller ObjectStructure}
	\begin{table}[H]
		{\rowcolors{2}{white}{gray!30}
			\begin{tabular}{|p{0,25\textwidth}|p{0,65\textwidth}|>{\centering}p{0,1\textwidth}|}
				\hline
				Testmethode & Testbeschreibung & Ergebnis\tabularnewline
				\hline
				\hspace{0pt}TestRegisterUser & Testet ob ein RegistrationRequest mit den richtigen Informationen über den Benutzer an den Server gesendet wird. &			\checkmark\tabularnewline
				\hspace{0pt}TestDeleteUser & Testet ob ein DeleteUserRequest mit der Device Id an den Server gesendet wird. & \checkmark\tabularnewline
				\hspace{0pt}TestCreateGroup & Testet ob ein CreateGroupRequest mit dem neuen Gruppennamen an den Server geschickt wird. & \checkmark\tabularnewline
				\hspace{0pt}TestJoinGroup & Testet ob ein JoinGroupRequest mit dem Link und der Device Id an den Server geschickt wird. & \checkmark\tabularnewline
				\hspace{0pt}TestDeleteGroup & Testet ob ein DeleteGroupRequest mit dem Gruppennamen an den Server geschickt wird. & \checkmark\tabularnewline
				\hspace{0pt}TestGetDeviceID & Testet ob eine gültige Device id vom Gerät abgerufen wird. & \checkmark\tabularnewline
				\hline
			\end{tabular}}
		\caption{AccountHandlerTest und GroupHandlerTest}
	\end{table}

	\paragraph{Client Model}
	\begin{table}[H]
		{\rowcolors{2}{white}{gray!30}
			\begin{tabular}{|p{0,25\textwidth}|p{0,65\textwidth}|>{\centering}p{0,1\textwidth}|}
				\hline
				Testmethode & Testbeschreibung & Ergebnis\tabularnewline
				\hline
				\hspace{0pt}TestOnCreate & Testet ob die Datenbank mit ihren zwei Tabellen erstellt wird. & \checkmark\tabularnewline
				\hspace{0pt}TestOnUpgrade & Testet ob die alte Datenbank gelöscht und dann eine neue aufgebaut wird. & \checkmark\tabularnewline
				\hspace{0pt}TestToSimpleAppointment & Testet ob ein Appointment erfolgreich zu einem SimpleAppointment gecastet wurde. & \checkmark\tabularnewline
				\hspace{0pt}TestSetAppointmentDate & Testet ob Datum und Uhrzeit für ein Treffen gesetzt wurden. & \checkmark\tabularnewline
				\hspace{0pt}TestSetDestination & Testet ob Zielort und GPS-Koordination für ein Treffen gesetzt wurden. & \checkmark\tabularnewline
				\hspace{0pt}TestCreateInviteLink & Testet ob ein CreateLinkRequest mit der Device Id und dem Gruppennamen an den Server geschickt wird. & \checkmark\tabularnewline
				\hspace{0pt}TestGetGroupUpdate & Testet ob ein GroupUpdateRequest mit der Device Id und dem Gruppennamen an den Server geschickt wird. & \checkmark\tabularnewline
				\hspace{0pt}TestMakeGroupMemberToAdmin & Testet ob ein MakeAdminRequest mit der Device Id, der Member Id und dem Gruppennamen an den Server geschickt wird. & \checkmark\tabularnewline
				\hspace{0pt}TestGetAllGroupMemberNames & Testet ob alle Gruppenmitglieder einer Gruppe aufgezählt werden. & \checkmark\tabularnewline
				\hspace{0pt}TestDeleteGroupMember & Testet ob ein KickMemberRequest mit der Device Id, der Member Id und dem Gruppennamen an den Server geschickt wird. & \checkmark\tabularnewline
				\hspace{0pt}TestLeaveGroup & Testet ob ein LeaveGroupRequest mit der Device Id und dem Gruppennamen an den Server geschickt wird. & \checkmark\tabularnewline
				\hspace{0pt}TestGetMemberTypeTrue & Testet ob für den Gruppenadministrator aus der Android Datenbank ausgelesen wird, dass er Admin ist. & \checkmark\tabularnewline
				\hspace{0pt}TestGetMemberTypeFalse & Testet ob für ein GruppenMitglied aus der Android Datenbank ausgelesen wird, dass es nicht Administrator ist. & \checkmark\tabularnewline
				\hspace{0pt}TestActivateGoService & Testet ob die Android Datenbank aktualisiert wird, sobald der GoStatus aktiviert wird. & \checkmark\tabularnewline
				\hspace{0pt}TestDeactivateGoService & Testet ob die Android Datenbank aktualisiert wird, sobald der GoStatus deaktiviert wird. & \checkmark\tabularnewline
				\hspace{0pt}TestChangeGroupName & Testet ob ein RenameGroupRequest mit der Device Id an den Server geschickt wird. & \checkmark\tabularnewline
				\hspace{0pt}TestLinkToString & Testet ob die vom Server erhaltenen Informationen für den Link zu einen String umgewandelt werden können. & \checkmark\tabularnewline
				\hline
			\end{tabular}}
			\caption{DBGroupHandler-, Appointment-, GroupClient und LinkTest}
		\end{table}



		\paragraph{Server Controller}
		\begin{table}[H]
			{\rowcolors{2}{white}{gray!30}
				\begin{tabular}{|p{0,25\textwidth}|p{0,65\textwidth}|>{\centering}p{0,1\textwidth}|}
					\hline
					Testmethode & Testbeschreibung & Ergebnis\tabularnewline
					\hline
					\hspace{0pt}TestGetUser & Testet ob man ein SimpleUser Objekt von der Datenbank bekommt. &	\checkmark\tabularnewline
					\hspace{0pt}TestGetGroup & Testet ob man ein GroupServer Objekt von der Datenbank bekommt. &	\checkmark\tabularnewline
					\hspace{0pt}TestPersistObject & Testet ob ein SimpleUser Objekt in der Datenbank gespeichert wird. &	\checkmark\tabularnewline
					\hspace{0pt}TestPersistObject & Testet ob ein GroupServer Objekt in der Datenbank gespeichert wird. &	\checkmark\tabularnewline
					\hspace{0pt}TestDeleteObject & Testet ob ein SimpleUser Objekt von der Datenbank gelöscht wurde. &	\checkmark\tabularnewline
					\hspace{0pt}TestDeleteObject & Testet ob ein GroupServer Objekt von der Datenbank gelöscht wurde. &	\checkmark\tabularnewline
					\hline
				\end{tabular}}
				\caption{ResourceManagerTest}
			\end{table}

		\paragraph{Server Model}
		\begin{table}[H]
			{\rowcolors{2}{white}{gray!30}
				\begin{tabular}{|p{0,25\textwidth}|p{0,65\textwidth}|>{\centering}p{0,1\textwidth}|}
					\hline
					Testmethode & Testbeschreibung & Ergebnis\tabularnewline
					\hline
					\hspace{0pt}TestAddAdmin & Testet ob ein Benutzer als Gruppenadministrator einer Gruppe hinzugefügt wird. &	\checkmark\tabularnewline
					\hspace{0pt}TestCreateLink & Testet ob ein Link erstellt wird und dieser auch die richtigen Informationen über die Gruppe und das Secret enthält. &	\checkmark\tabularnewline
					\hspace{0pt}TestJoinGroup & Testet ob ein Benutzer einer Gruppe über einen Link betreten kann und wenn der Link nochmal verwendet wird, dass das nicht geht. &	\checkmark\tabularnewline
					\hspace{0pt}TestRemoveMember & Testet ob ein Gruppenmitglied aus einer Gruppe entfernt wird. &	\checkmark\tabularnewline
					\hspace{0pt}TestGetGpsData & Testet ob die GPS Daten für einen Benutzer gesetzt werden. &	\checkmark\tabularnewline
					\hspace{0pt}TestGetMemberAssociations & Testet ob man alle Informationen über den Status der Gruppenmitglieder erhält. &	\checkmark\tabularnewline
                         \hspace{0pt}TestCopy & Testet ob Attribute einer Gruppe korrekt kopiert werden. &	\checkmark\tabularnewline
                         \hspace{0pt}TestGetSerializableMemberList & Testet ob Mitglieder Informationen korrekt als Liste von SerializableMember zurückgegeben werden. &	\checkmark\tabularnewline
					\hline
				\end{tabular}}
				\caption{GroupServerTest}
			\end{table}

		\paragraph{Server Communication}
		\begin{table}[H]
			{\rowcolors{2}{white}{gray!30}
				\begin{tabular}{|p{0,25\textwidth}|p{0,65\textwidth}|>{\centering}p{0,1\textwidth}|}
					\hline
					Testmethode & Testbeschreibung & Ergebnis\tabularnewline
					\hline
					\hspace{0pt}TestSetup & Testet ob lokale Variablen initialisiert wurden. &	\checkmark\tabularnewline
					\hspace{0pt}TestTearDown & Testet ob lokale Variablen zurückgesetzt wurden. &	\checkmark\tabularnewline
					\hspace{0pt}TestGetRequest & Testet die Antwort einer HTTP-GET Request und ob der Server erreichbar ist. &	\checkmark\tabularnewline
					\hspace{0pt}TestRegistration & Testet ob eine Benutzer-erstellen Anfrage einen Benutzer auf dem Server erzeugt. &	\checkmark\tabularnewline
					\hspace{0pt}TestRecoverAccount & Testet ob man seine Benutzer-ID zurückerhält, wenn man sich erneut registriert. &	\checkmark\tabularnewline
					\hspace{0pt}TestDeleteUser & Testet ob eine DeletUserRequest einen Benutzer auf dem Server löscht. &	\checkmark\tabularnewline
					\hspace{0pt}TestRenameUser & Testet ob ein Benutzer auf dem Server mit einer RenameUserRequest umbenannt wird. &	\checkmark\tabularnewline
					\hspace{0pt}TestRenameUserFail & Testet ob ein Benutzer auf dem Server mit einer RenameUserRequest nicht umbenannt wird falls er nicht registriert ist. &	\checkmark\tabularnewline
					\hspace{0pt}TestCreateGroup & Testet ob nach einer CreateGroupRequest eine Gruppe auf dem Server erzeugt wurde. &	\checkmark\tabularnewline
					\hspace{0pt}TestCreateGroupFail & Testet ob das Erstellen einer Gruppe mit gleichem Namen scheitert. &	\checkmark\tabularnewline
					\hspace{0pt}TestDeleteGroup & Testet ob eine Gruppe nach einer DeleteGroupRequest vom Server gelöscht wurde. &	\checkmark\tabularnewline
					\hspace{0pt}TestDeleteGroupFail & Testet ob eine Gruppe nach einer DeleteGroupRequest nicht vom Server gelöscht wurde falls die Gruppe auf dem Server nicht existiert. &	\checkmark\tabularnewline
					\hspace{0pt}TestGroupNameAvailable & Testet ob ein Gruppenname wieder verfügbar ist, nachdem die Gruppe gelöscht wurde. &	\checkmark\tabularnewline
					\hspace{0pt}TestCreateLink & Testet ob durch eine CreateLinkRequest ein Link und ein "Secret" auf dem Server erzeugt wurden. &	\checkmark\tabularnewline
                         \hspace{0pt}TestJoinGroup & Testet ob ein Benutzer mit Hilfe eines Links einer Gruppe auf dem Server beitreten kann (JoinGroupRequest). &	\checkmark\tabularnewline
					\hspace{0pt}TestKickMember & Testet ob ein Gruppenmitglied von einem Admin aus der Gruppe geschmissen wird.  &	\checkmark\tabularnewline
					\hspace{0pt}TestLeaveGroup & Testet ob ein Benutzer nach einer LeaveGroupRequest aus der Gruppe gelöscht wird. &	\checkmark\tabularnewline
					\hspace{0pt}TestMakeAdmin & Testet ob ein Gruppenmitglied nach einer MakeAdminRequest Administratorrechte in der Gruppe hat. &	\checkmark\tabularnewline
					\hspace{0pt}TestSetAppointment & Testet ob durch eine SetAppointmentRequest ein Treffen in der Gruppe auf dem Server festgelegt wird. &	\checkmark\tabularnewline
					\hspace{0pt}TestBroadcastGps & Testet ob GPS-Daten korrekt an den Server gesendet und wieder empfangen werden. &	\checkmark\tabularnewline
                         \hspace{0pt}TestUpdate & Testet ob Gruppendaten vom Server abgefragt werden können (UpdateRequest). &	\checkmark\tabularnewline

					\hline
				\end{tabular}}
				\caption{RequestTest}
			\end{table}


\subsection{Android Tests}
	\paragraph{Client Controller Database}
	\begin{table}[H]
		{\rowcolors{2}{white}{gray!30}
			\begin{tabular}{|p{0,25\textwidth}|p{0,65\textwidth}|>{\centering}p{0,1\textwidth}|}
				\hline
				Testmethode & Testbeschreibung & Ergebnis\tabularnewline
				\hline
				\hspace{0pt}TestInsertNewGroup & Testet ob eine neue Gruppe in die Gruppentabelle der Android Datenbank hinzugefügt wird. & \checkmark\tabularnewline
				\hspace{0pt}TestReadOneGroupRow & Testet ob über den Gruppennamen die richtige Gruppe aus der Gruppentabelle der Android Datenbank ausgelesen wird. & \checkmark\tabularnewline
				\hspace{0pt}TestReadAllGroupNames & Testet ob alle Gruppennamen aus der Gruppentabelle der Android Datenbank ausgelesen werden. & \checkmark\tabularnewline
				\hspace{0pt}TestDeleteAllGroups & Testet ob alle Gruppen aus der Gruppentabelle der Android Datenbank gelöscht werden. & \checkmark\tabularnewline
				\hspace{0pt}TestDeleteOneGroupRow & Testet ob eine Gruppe aus der Gruppentabelle der Android Datenbank gelöscht wird. & \checkmark\tabularnewline
				\hspace{0pt}TestUpdateGroupDate & Testet ob der Eintrag einer Gruppe in der Gruppentabelle der Android Datenbank angepasst wird, sobald sich das Appointment, der GoStatus oder der Gruppenname ändert. & \checkmark\tabularnewline
				\hspace{0pt}TestInserUserData & Testet ob ein neues Mitglied einer Gruppe der Benutzertabelle der Android Datenbank hinzugefügt wird. & \checkmark\tabularnewline
				\hspace{0pt}TestReadAllGroupMembers & Testet ob alle Mitgliedernamen einer Gruppe aus der Benutzertabelle der Android Datenbank ausgelesen werden. & \checkmark\tabularnewline
				\hspace{0pt}TestReadAllGroupMemberIds & Testet ob alle Mitglieder-Id's einer Gruppe aus der Benutzertabelle der Android Datenbank ausgelesen werden. & \checkmark\tabularnewline
				\hspace{0pt}TestReadAdminOrMemberStatus & Testet ob der richtige Admin oder Member Status aus der Benutzertabelle der Android Datenbank eines Mitglieds zu einer Gruppe ausgelesen wird. & \checkmark\tabularnewline
				\hspace{0pt}TestDeleteAllUserAndGroups & Testet ob alle Mitglieder und zugehörigen Gruppen aus der Benutzertabelle der Android Datenbank gelöscht werden. & \checkmark\tabularnewline
				\hspace{0pt}TestDeleteUserFromGroup & Testet ob ein Mitglied einer Gruppe aus der Benutzertabelle der Android Datenbank gelöscht wird. & \checkmark\tabularnewline
				\hspace{0pt}TestDeleteAllGroupMembers & Testet ob alle Mitglieder einer Gruppe aus der Benutzertabelle der Android Datenbank gelöscht werden. & \checkmark\tabularnewline
				\hspace{0pt}TestUpdateGroupNameInAlloc & Testet ob der Gruppenname einer Gruppe für alle Mitglieder in der Benutzertabelle der Android Datenbank angepasst wird, wenn dieser geändert wurde.. & \checkmark\tabularnewline
				\hspace{0pt}TestUpdateUserName & Testet ob der Benutzername vom Aktuellen Benutzer in allen Gruppen in denen er Mitglied ist, in der Benutzertabelle der Android Datenbank angepasst wird, wenn er seinen Benutzernamen ändert. & \checkmark\tabularnewline
				\hline
			\end{tabular}}
			\caption{GroupServiceTest und UserServiceTest}
		\end{table}
		\begin{table}[H]
			{\rowcolors{2}{white}{gray!30}
			\begin{tabular}{|p{0,25\textwidth}|p{0,65\textwidth}|>{\centering}p{0,1\textwidth}|}						\hline
				Testmethode & Testbeschreibung & Ergebnis\tabularnewline
				\hline
				\hspace{0pt}TestUpdateGroupMemberToAdmin & Testet ob ein Gruppenmitglied in der Benutzertabelle der Android Datenbank zu einem Administrator derjenigen Gruppe gemacht wird. & \checkmark\tabularnewline
				\hline
			\end{tabular}}
			\caption{GroupServiceTest und UserServiceTest Fortsetzung}
		\end{table}

\subsection{Integration Tests}
	\begin{table}[H]
		{\rowcolors{2}{white}{gray!30}
			\begin{tabular}{|p{0,1\textwidth}||p{0,2\textwidth}|p{0,65\textwidth}|>{\centering}p{0,1\textwidth}|}
				\hline
				Testfall &Testmethode & Testbeschreibung & Ergebnis\tabularnewline
				\hline
				\hspace{0pt}/T010/& Benutzer registrieren & Wenn der Benutzer mit dem Internet verbunden ist und der gewählte Benutzernamen gültig ist, wird er auf dem Server angelegt und danach in den Shared Preferences gespeichert. & \checkmark\tabularnewline
				\hspace{0pt}/T020/& Benutzer registrieren Fehlschlag 1 & Wenn der Benutzer mit dem Internet verbunden ist aber der gewählte Benutzernamen ungültig ist, wird er nicht auf dem Server angelegt und somit auch nicht in den Shared Preferences. & \checkmark\tabularnewline
				\hspace{0pt}/T030/& Benutzer registrieren Fehlschlag 2 & Wenn der Benutzer nicht mit dem Internet verbunden ist wird er nicht auf dem Server angelegt und somit auch nicht in den Shared Preferences. & \checkmark\tabularnewline
				\hspace{0pt}/T040/& Benutzeraccount löschen & Wenn der Benutzer mit dem Internet verbunden ist und auswählt seinen Account zu löschen, dann wird dieser vom Server gelöscht als auch alle seine Daten über ihn und seine Gruppen von der Android Datenbank. & \checkmark\tabularnewline
				\hspace{0pt}/T050/& Benutzeraccount löschen Fehlschlag & Wenn der Benutzer nicht mit dem Internet verbunden ist, wird er weder vom Server noch von der Android Datenbank gelöscht. & \checkmark\tabularnewline
				\hspace{0pt}/T060/& Gruppe erstellen & Wenn der Benutzer mit dem Internet verbunden und der Gruppenname gültig ist, dann wird die Gruppe mit ihm als einziges Mitglied (als Administrator) auf dem Server und in der Android Datenbank angelegt. & \checkmark\tabularnewline
				\hspace{0pt}/T070/& Gruppe erstellen Fehlschlag & Wenn der Benutzer nicht mit dem Internet verbunden ist und eine Gruppe erstellt, dann wird diese weder auf dem Server noch auf der Android Datenbank angelegt. & \checkmark\tabularnewline
				\hspace{0pt}/T080/& Gruppe löschen & Wenn der Benutzer mit dem Internet verbunden ist und auswählt eine Gruppe zu löschen, dann wird diese und alle ihre Mitglieder vom Server als auch von der Android Datenbank gelöscht. & \checkmark\tabularnewline
				\hspace{0pt}/T090/& Gruppe löschen Fehlschlag & Wenn der Benutzer nicht mit dem Internet verbunden ist, wird die Gruppe und ihre Mitglieder weder vom Server noch von der Android Datenbank gelöscht. & \checkmark\tabularnewline
				\hspace{0pt}/T100/& Gruppenlink erstellen & Wenn der Benutzer Administrator und mit dem Internet verbunden ist und ein Mitglied einladen möchte, dann erhält er einen Link, welcher auf dem Server in Zusammenhang mit der jeweiligen Gruppe gespeichert wird. & \checkmark\tabularnewline
				\hspace{0pt}/T110/& Gruppenlink erstellen Fehlschlag & Wenn der Benutzer Administrator aber nicht mit dem Internet verbunden ist, dann erhält er keinen Link und es wird auch keiner in Zusammenhang mit der Gruppe auf dem Server gespeichert. & \checkmark\tabularnewline
				\hline
			\end{tabular}}
			\caption{Testfälle aus dem Pflichtenheft}
		\end{table}
	\begin{table}[H]
		{\rowcolors{2}{white}{gray!30}
			\begin{tabular}{|p{0,1\textwidth}||p{0,2\textwidth}|p{0,65\textwidth}|>{\centering}p{0,1\textwidth}|}
				\hline
				Testfall &Testmethode & Testbeschreibung & Ergebnis\tabularnewline
				\hline
				\hspace{0pt}/T120/& Gruppe über Link beitreten & Wenn der Benutzer mit dem Internet verbunden und bereits bei der Go-App registriert ist, dann wird der Link mit der App geöffnet, der Benutzer der Gruppe hinzugefügt, auf dem Server und der Android Datenbank gespeichert und der Link vom Server gelöscht. & \checkmark\tabularnewline
				\hspace{0pt}/T130/& Gruppe über Link beitreten Fehlschlag & Wenn der Benutzer bereits bei der Go-App registriert aber nicht mit dem Internet verbunden ist, dann wird der Benutzer der Gruppe nicht hinzugefügt und der Link nicht vom Server gelöscht. & \checkmark\tabularnewline
				\hspace{0pt}/T140/ & Gruppe verlassen & Wenn der Benutzer mit dem Internet verbunden ist und auswählt eine Gruppe zu verlassen, dann wird er auch als Mitglied dieser Gruppe vom Server und die gesamte Gruppe von der Android Datenbank gelöscht. & \checkmark\tabularnewline
				\hspace{0pt}/T150/ & Gruppe verlassen Fehlschlag & Wenn der Benutzer nicht mit dem Internet verbunden ist und auswählt eine Gruppe zu verlassen, dann wird er nicht als Mitglied dieser Gruppe vom Server und der Android Datenbank gelöscht. & \checkmark\tabularnewline
				\hspace{0pt}/T160/ & Mitglied aus Gruppe entfernen & Wenn der Benutzer Administrator und mit dem Internet verbunden ist und auswählt ein Mitglied zu löschen, dann wird dieses aus der Gruppe entfernt und vom Server als auch der Android Datenbank gelöscht. & \checkmark\tabularnewline
				\hspace{0pt}/T170/ & Mitglied aus Gruppe entfernen Fehlschlag & Wenn der Benutzer Administrator aber nicht mit dem Internet verbunden ist und auswählt ein Mitglied zu löschen, dann wird dieses nicht der Gruppe entfernt und auch weder vom Server noch von der Android Datenbank gelöscht. & \checkmark\tabularnewline
				\hspace{0pt}/T180/ & Zielort festlegen & Wenn der Benutzer Administrator, mit dem Internet verbunden ist und einen Zielort auswählt, dann wird der Zielort der Gruppe auf dem Server und der Android Datenbank aktualisiert. & \checkmark\tabularnewline
				\hspace{0pt}/T190/ & Uhrzeit festlegen & Wenn der Benutzer Administrator, mit dem Internet verbunden ist und eine Uhrzeit für den Treffpunkt wählt, dann wird die Uhrzeit der Gruppe auf dem Server und der Android Datenbank aktualisiert. & \checkmark\tabularnewline
				\hspace{0pt}/T200/ & Zielort/ Uhrzeit festlegen Fehlschlag & Wenn der Benutzer Administrator, nicht mit dem Internet verbunden ist und Zielort und/oder Uhrzeit auswählt, dann wird die Gruppe weder auf dem Server noch in der Android Datenbank aktualisiert. & \checkmark\tabularnewline
				\hspace{0pt}/T210/ & Go-Button drücken & Wenn der Benutzer mit dem Internet verbunden ist und den Go-Button für eine Gruppe drückt, dann wird dies in der Android Datenbank angepasst und sein Standort alle 15 Sekunden auf dem Server aktualisiert und an die anderen Gruppenmitglieder weitergeleitet.  & \checkmark\tabularnewline
				\hline
			\end{tabular}}
			\caption{Fortsetzung Testfälle aus dem Pflichtenheft}
	\end{table}
	\begin{table}[H]
		{\rowcolors{2}{white}{gray!30}
			\begin{tabular}{|p{0,1\textwidth}||p{0,2\textwidth}|p{0,65\textwidth}|>{\centering}p{0,1\textwidth}|}
				\hline
				Testfall &Testmethode & Testbeschreibung & Ergebnis\tabularnewline
				\hline
				\hspace{0pt}/T220/ & Go-Button drücken Fehlschlag & Wenn der Benutzer nicht  mit dem Internet verbunden ist und den Go-Button drückt, dann wird dieser weder in der Android Datenbank angepasst noch sein Standort an den Server übermittelt und dort gespeichert. & \checkmark\tabularnewline
				\hspace{0pt}/T230/ & GPS-Daten auf Karte anzeigen & Wenn der Benutzer den Go-Button aktiviert und mit dem Internet verbunden ist, dann erhält er alle 15 Sekunden die Standorte der anderen Gruppenmitglieder. Diese werden auf der Karte zu einem Punkt zusammengefasst, wenn sie nahe beieinander liegen. & \checkmark\tabularnewline
				\hspace{0pt}/T240/ & Verschlüsselung zwischen Client und Server & Wenn der Benutzer mit dem Internet verbunden ist und Daten an den Server übermittelt/ empfängt, dann sind diese verschlüsselt. & \checkmark\tabularnewline
				\hspace{0pt}/T250/ & Go-Button deaktivieren & Wenn der Benutzer seinen Go-Button nochmal drückt nachdem er diesen aktiviert hat, dann wird sein Standort nicht mehr an den Server übermittelt und die Android Datenbank aktualisiert.& \checkmark\tabularnewline
				\hspace{0pt}/T260/ & Gruppenparameter aktualisieren & Wenn der Benutzer auf eine Gruppe drückt, dann wird die Gruppe, also ihre Mitglieder und der Treffpunkt auf dem Server und der Android Datenbank aktualisiert. & \checkmark\tabularnewline
				\hspace{0pt}/T270/ & Gruppenparameter aktualisieren Fehlschlag & Wenn der Benutzer auf eine Gruppe drückt aber nicht mit dem Internet verbunden ist, dann werden die Gruppenparameter weder in der Android Datenbank noch auf dem server aktualisiert. & \checkmark\tabularnewline
				\hline
			\end{tabular}}
			\caption{Zweite Fortsetzung der Testfälle aus dem Pflichtenheft}
	\end{table}



